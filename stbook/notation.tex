\stbkpreface[Notation and typography]

\subsubsection*{Math typesetting conventions and custom commands (to be removed later)}

The following lists contain some notational conventions for equations and some 
custom commands. The conventions are guided by what is typically done in Stata Press books.
\begin{itemize}[itemsep=0pt,topsep=0pt]
    \item omit punctuation in display-style equations
    \item use small caps for acronyms of estimators and other definitions, 
       such as \stsc{IF}, \stsc{ARE}, \stsc{ASV}, \stsc{BIC}, \stsc{MM} estimator, \stsc{S} estimator, etc.
    \item use boldface lowercase letters for vectors ($\stvec{x}$)
    \item use boldface uppercase letters for matrices ($\stmat{X}$)
    \item use $E(X)$ for expectation
    \item use $\mathrm{Var}(X)$ and $\mathrm{Cov}(X,Y)$ for 
          variance and covariance: \verb+\mathrm{Var}+ \dots
    \item use $\mathcal{N}(a, b)$ for normal distribution: \verb+\mathcal{N}+ \dots
    \item use $\Pr(x<y)$ for probability: \verb+\Pr+ \dots
    \item use $'$ for both transposition $\stvec{x}'$ (\verb+\stvec{x}'+) and first derivative $F'(x)$ (\verb+F'(x)+)
    \item use of parentheses
    \begin{itemize}
        \item parentheses (): grouping/order of operations, functions, open intervals
        \item square brackets []: matrices, closed intervals
        \item nested [()] for grouping/order of operations: may use square brackets as second set of parentheses 
        for visual distinction
        \item braces \{\}: sets
    \end{itemize}
    \item commands:
        \begin{tabular}[t]{lll}
            small caps acronyms      & \verb+\stsc{IQR}(X)+       & $\stsc{IQR}(X)$         \\
            estimation hat           & \verb+\sthat{\theta}+      & $\sthat{\theta}$     \\
            mean bar                 & \verb+\stbar{x}+           & $\stbar{x}$          \\
            vectors                  & \verb+\stvec{x}+           & $\stvec{x}$          \\
            matrices                 & \verb+\stmat{X}+           & $\stmat{X}$          \\
            bold symbols             & \verb+\boldsymbol\theta+   & $\boldsymbol\theta$  \\
            integral dif operator    & \verb+\int_a^b x \dif F(x)+& $\int_a^b x \dif F(x)$ \\
            indicator function       & \verb+\I(x<X)+             & $\I(x<X)$            \\
            sign operator            & \verb+\sign(x)+            & $\sign(x)$            \\
            median operator          & \verb+\med(x)+             & $\med(x)$            \\
        \end{tabular}
\end{itemize}



\subsubsection*{Stata code, datasets, programs, and references to manuals}

In this book we assume that you are somewhat familiar with \stata, that you
know how to input data and to use previously created datasets, create new
variables, run regressions, and the like. Generally, we use the \stcmd{typewriter font}
to refer to Stata commands, syntax, and variables. A “dot”
prompt followed by a command indicates that you can type verbatim what is
displayed after the dot (in context) to replicate the results in the book.

The data we use in this book are freely available for you to download, using a
net-aware Stata, from the Stata Press website,
\url{http://www.stata-press.com}. In fact, when we introduce new datasets, we
merely load them into Stata the same way that you would. For example,

\begin{stlog}
. use http://www.stata-press.com/data/!!!/football.dta, clear
\end{stlog}

\noindent
In addition, the Stata packages presented in this book may be obtained by typing

\begin{stlog}
. ssc install robstat
\oom
. ssc install robreg
\oom
. ssc install robmv
\oom
\end{stlog}

\alert{Also say what other packages need to be installed (if any), e.g. \stcmd{moremata}, I think.}

Throughout the book, we often refer to the Stata manuals using \rref{\!}, \pref{\!}, etc.
For example, \rref{regress} refers to the \textsl{Stata Reference Manual} entry for
\stcmd{regress}, and \pref{matrix} refers to the entry for \stcmd{matrix} in
the \textsl{Stata Programming Manual}.

\subsubsection*{Mathematical and statistical symbols}

We also assume that you have basic knowledge of mathematics and statistics,
although we tried to keep the exposition as simple and non-technical as
possible. Below is a list of some mathematical and statistical symbols that we
will frequently use in the book.

\begin{description}[leftmargin=6em,style=nextline]
    \item[$X, Y, Z, \dots$]
        random variables
    \item[$x_i, y_i, z_i, \dots$]
        realizations (observations) of random variables
    \item[$n$]
        number of observations
    \item[$x_{(i)}$]
        $i$th order statistic of $x_1, \dots, x_n$ ($i$th observation in the list of observations sorted in ascending order)
    \item[$F(x)$]
        cumulative distribution function of a random variable; \dots
    \item[$f(x)$]
        density \dots
    \item[$F'(x)$]
        first derivative of function $F(x)$, that is $F'(x) = dF(x)/dx = f(x)$;
        we use $'$ for both the first derivative of a function and the
        transposition of a vector or matrix
    \item[$\mathcal{N}(\mu, \sigma)$]
        normal distribution with mean $\mu$ and standard deviation $\sigma$
    \item[$\mathcal{N}(0, 1)$]
        standard normal distribution
    \item[$|x|$]
        absolute value of $x$
    \item[$\Vert\stvec{x}\Vert$]
        Euclidean norm of vector $\stvec{x} = (x_1, \dots, x_p)'$, that is, $\Vert\stvec{x}\Vert = \sqrt{x_1^2 + \dots + x_p^2}$
    \item[$\lceil x \rceil$]
        smallest integer greater or equal to $x$
    \item[$\lfloor x \rfloor$] 
        largest integer smaller or equal to $x$
    \item[$\stvec{x}'$, $\stmat{X}'$]
        transposition of a vector or a matrix; we use $'$ for both
        the transposition of a vector or matrix and the first derivative of a
        function
    \item[\normalfont i.i.d.]
        independent and identically distributed
    \item[$\displaystyle\lim_{x\rightarrow y} g(x)$]
        limiting value of function $g(x)$ as $x$ approaches $y$
    \item[$\displaystyle\sup_x g(x)$]
        supremum (least upper bound) of function $g(x)$ with respect 
        to argument $x$
    \item[$\sign(x)$]
        the sign of $x$; to be precise, $\sign(x)=-1$ if $x<0$, $\sign(x)=+1$ if $x>0$, $\sign(x)=0$ if $x=0$
    \item[$X\sim F$]
        random variable $X$ is distributed as $F$
    \item[$X\approx F$]
        random variable $X$ is approximately distributed as $F$
    \item[\alert{...}]
        \alert{...}
\end{description}




\endinput
